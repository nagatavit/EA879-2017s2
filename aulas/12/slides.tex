\documentclass{beamer}
%
% Choose how your presentation looks.
%
% For more themes, color themes and font themes, see:
% http://deic.uab.es/~iblanes/beamer_gallery/index_by_theme.html
%
\mode<presentation>
{
  \usetheme{Madrid}      % or try Darmstadt, Madrid, Warsaw, ...
  \usecolortheme{default} % or try albatross, beaver, crane, ...
  \usefonttheme{default}  % or try serif, structurebold, ...
  \setbeamertemplate{navigation symbols}{}
  \setbeamertemplate{caption}[numbered]
}

\usepackage[english]{babel}
\usepackage[utf8x]{inputenc}
\usepackage{graphicx}
\usepackage{array}

\title[12-posix-e-processos]{EA879 -- Introdução ao Software
Básico\\Processos e POSIX}
\author{Tiago F. Tavares}
\institute{FEEC -- UNICAMP}
\date{Aula 12 -- 15/agosto/2017}

\begin{document}

\begin{frame}
  \titlepage
\end{frame}

% Uncomment these lines for an automatically generated outline.
%\begin{frame}{Outline}
%  \tableofcontents
%\end{frame}

\section{Introdução}

\begin{frame}{Objetivos}
  \Large
  \begin{itemize}
    \item Entender como sistemas POSIX administram os processos que executam na
      máquina.
    \item Entender como implementar sistemas multiprocesso em C.
  \end{itemize}
\end{frame}


\begin{frame}[fragile]{Os contadores de letras}
  \centering
  \Large
  Um gerente tem um time de estagiários (20 deles)
  numa equipe que deve contar as letras existentes em 30 livros.
  Como seria uma boa
  forma de organizar a equipe de forma a acelerar ao máximo o processo de
  contagem, sabendo que cada estagiário conta letras numa velocidade diferente??
\end{frame}

\begin{frame}[fragile]{Os contadores de letras}
  \centering
  \large
  Um gerente tem um time de estagiários (20 deles)
  numa equipe que deve contar as letras existentes em 30 livros.
  Como seria uma boa
  forma de organizar a equipe de forma a acelerar ao máximo o processo de
  contagem, sabendo que cada estagiário conta letras numa velocidade diferente?

  \begin{enumerate}
    \item Quantas tarefas existem?
    \item Quantos trabalhadores são necessários? Por que?
    \item Como é possível coordenar o trabalho, mantendo registro de quem já
      terminou suas tarefas
  \end{enumerate}
\end{frame}

\begin{frame}[fragile]{Análise de código}
  \centering
  \large
  A instrução \textsc{fork()} divide o processo que a chama em um processo pai e
  um processo filho. Assim, é possível criar duas linhas de execução isoladas
  em uma mesma base de código. Ver: multiprocesso.c.
\end{frame}

\begin{frame}[fragile]{Segmentação de memória}
  \centering
  \Large
  Por que a alteração de uma variável no processo-filho não tem impacto na
  variável do processo-pai e vice-versa?
\end{frame}

\begin{frame}[fragile]{Programas responsivos}
  \centering
  \large
  Um programa que baixa vídeos da Internet funciona da seguinte forma:
  \begin{enumerate}
    \item Espera usuário clicar um botão e verifica qual é
    \item Se o botão é o baixar
      \begin{enumerate}
        \item Conecta-se com servidor
        \item Baixa vídeo
        \item Grava vídeo em um local permanente no computador
        \item Volta para o primeiro passo
      \end{enumerate}
     \item Se o botão é de terminar, encerra programa
  \end{enumerate}

  Considerando somente esse fluxo de execução, o que acontece se o usuário
  clicar o botão para baixar vídeos e, logo em seguida (antes do vídeo terminar
  de baixar), tentar encerrar o programa?
\end{frame}

\begin{frame}[fragile]{Programas responsivos}
  \centering
  \large
  \begin{enumerate}
    \item Espera usuário clicar um botão e verifica qual é
    \item Se o botão é o baixar
      \begin{enumerate}
        \item Conecta-se com servidor
        \item Baixa vídeo
        \item Grava vídeo em um local permanente no computador
        \item Volta para o primeiro passo
      \end{enumerate}
     \item Se o botão é de terminar, encerra programa
  \end{enumerate}

  Quais dessas instruções poderiam ser colocadas em um processo-filho, de forma
  que a interface com o usuário continuasse funcionando durante o download? Ou,
  em C: onde deveria entrar uma instrução \textsc{fork()}?
\end{frame}

\begin{frame}[fragile]{Interface separada de implementação}
  \centering
  \Large
  Analise de código: interfacemultiprocesso.c.
\end{frame}

\begin{frame}[fragile]{Tabela de processos POSIX}
  \centering
  \Large
  Em POSIX, processos são identificados por um identificador chamado Process
  ID (ou, em português, identificador de processo). Os PIDs dos processos em
  execução são guardados numa tabela que pode ser acessada usando o comando
  \textsc{ps}.

  Se você decidir encerrar um processo, pode usar o comando \textsc{kill [pid]}
  para informá-lo que deve ser encerrado.
\end{frame}

\begin{frame}[fragile]{Posicionamento}
  \centering
  \large
  Associe a sistemas de tempo real em um microcontrolador usando scheduler
  ou a sistemas multi-processo:
  \begin{enumerate}
    \item Uma tarefa pode durar quanto tempo quiser sem que isso tenha um
      impacto grande nas outras tarefas;
    \item Todas as tarefas compartilham as mesmas variáveis e espaços de memória;
    \item Os programadores devem estar cientes das outras tarefas que também
      executarão no sistema;
    \item Uma tarefa pode ser interrompida para que outra tarefa seja executada;
    \item É possível criar novas tarefas à medida que novos pedidos vão
      chegando;
    \item O tempo de execução de uma tarefa é previsível
  \end{enumerate}
\end{frame}




\end{document}
