\documentclass{beamer}
%
% Choose how your presentation looks.
%
% For more themes, color themes and font themes, see:
% http://deic.uab.es/~iblanes/beamer_gallery/index_by_theme.html
%
\mode<presentation>
{
  \usetheme{Madrid}      % or try Darmstadt, Madrid, Warsaw, ...
  \usecolortheme{default} % or try albatross, beaver, crane, ...
  \usefonttheme{default}  % or try serif, structurebold, ...
  \setbeamertemplate{navigation symbols}{}
  \setbeamertemplate{caption}[numbered]
}

\usepackage[english]{babel}
\usepackage[utf8x]{inputenc}
\usepackage{graphicx}
\usepackage{array}

\title[14-memshare]{EA879 -- Introdução ao Software
Básico\\Estrutura Dispatcher}
\author{Tiago F. Tavares}
\institute{FEEC -- UNICAMP}
\date{Aula 14 -- 10/outubro/2018}

\begin{document}

\begin{frame}
  \titlepage
\end{frame}

% Uncomment these lines for an automatically generated outline.
%\begin{frame}{Outline}
%  \tableofcontents
%\end{frame}

\section{Introdução}

\begin{frame}{Objetivos}
  \Large
  \begin{itemize}
    \item Entender como é possível compartilhar memória entre dois processos
    \item Entender quanto a estrutura de dispatcher é útil
    \item Comparar a estrutura de memória compartilhada com a estrutura pipes
  \end{itemize}
\end{frame}


\begin{frame}[fragile]{Revisão}
  \centering
  \large
  \begin{verbatim}
  #include <stdio.h>
  int main() {
    char x='a';
    while ( (x>='a') && (x<='z') ) {
      scanf("%c", &x);
      if (x=='c') printf("cc");
      else printf("%c", x);
    }
  }
  \end{verbatim}

  Qual será a saída da chamada:
  \begin{verbatim}
  echo "bandeco" | ./main | ./main | ./main
  \end{verbatim}
\end{frame}


\begin{frame}[fragile]{Revisão}
  \centering
  \Large
  Assinale as alternativas corretas:
  \begin{enumerate}
    \item Cada chamada ao programa ./main leva à criação de um novo processo,
      cada um com seu próprio identificador PID
    \item O programador que invocou processos usando pipes deve saber exatamente
      como cada um dos programas funciona internamente
    \item O programador que invocou processos usando pipes só precisa saber
      quais são as entradas e saídas que o programa recebe ou fornece
  \end{enumerate}
\end{frame}

\begin{frame}[fragile]{Problema: pizzaria delivery}
  \centering
  \large
  Uma pizzaria faz entregas à domicílio. Estime o tempo médio que leva entre
  receber a
  ligação de um cliente e a entrega da pizza, pensando que o tempo de preparo de
  uma pizza é muito pequeno, e que tanto a pizzaria quanto os clientes ficam
  em Barão
  Geraldo (região central, entre a moradia e a Unicamp), e que, ao sair para
  entrega, só é possível levar uma pizza de cada vez, assumindo os casos:
  \begin{enumerate}
    \item A pizzaria tem um único funcionário, que atende o telefone, faz pizzas
      e entrega;
    \item A pizzaria tem três funcionários: um que atende o telefone, um que faz
      pizzas e outro que faz entregas;
    \item A pizzaria tem dez funcionários: um que atende o telefone, um que
      faz pizzas e oito entregadores;
  \end{enumerate}
  Considere os casos de dias de baixa demanda e de altíssima demanda.
\end{frame}

\begin{frame}[fragile]{Problema: servidor responsivo}
  \centering
  \large
  Um servidor web recebe requisições de termos de busca, e retorna listas de
  documentos que são relevantes a essa busca. Fazer uma busca no banco de dados
  demora alguns segundos, e, tipicamente, o processo de envio demora mais alguns
  segundos, dependendo da velocidade da conexão do usuário. Em momentos de baixa
  e de alta demanda, qual é o tempo de resposta médio do servidor para os casos:
  \begin{enumerate}
    \item O servidor opera em um único processo que recebe requisições, faz a
      busca no banco de dados e retorna os resultados ao usuário;
    \item Ao receber uma nova requisição, o servidor cria um novo processo-filho
      (usando fork()) que faz a busca no banco de dados e retorna documentos ao
      usuário.
  \end{enumerate}
\end{frame}

\begin{frame}[fragile]{Dispatcher}
  \centering
  \Large
  Trata-se de um \textit{design pattern} no qual um elemento central é
  responsável por inicializar \textit{trabalhadores} para cada uma das
  requisições que recebe.
\end{frame}

\begin{frame}[fragile]{Dispatcher ou pipeline?}
  \centering
  \large
  Cada afirmação abaixo está mais relacionada ao padrão dispatcher ou a idéia de
  pipelines?
  \begin{enumerate}
  \item Cada programa do padrão POSIX faz somente uma coisa, e faz bem.
  \item Os gerenciadores de plugin do Firefox executam em processos isolados.
  \item O WhatsApp permite manter diversas conversas abertas sem que uma
    influencie a outra.
  \item É possível compilar um documento LaTeX para PDF, convertê-lo para
    texto puro e então contar suas palavras.
  \item Cada trabalhador numa linha de produção executa uma pequena parte do
    processo produtivo.
  \item Num ateliê de artesãos, cada um é responsável por fazer uma das peças
    que vão decorar o casamento da Marina Rui Barbosa.
  \end{enumerate}
\end{frame}

\begin{frame}[fragile]{Problema: multiplicação de matrizes}
  \centering
  \large
  Junto ao seu grupo, analise o problema da multiplicação de matrizes:
  \begin{equation}
  X = BA \rightarrow x_{i,j} = \sum_{k=1}^K b_{i, k} a_{k, j}.
  \end{equation}
  \begin{enumerate}
    \item O cálculo de cada elemento de $X$ depende do resultado dos outros
      elementos?
    \item O cálculo de cada elemento de $X$ poderia ser executado em paralelo
      aos demais?
    \item Para resolver computacionalmente esse problema, é mais adequado o
      padrão dispatcher ou o padrão pipeline?
    \item Que problema (sem considerar o desempenho) será encontrado se um novo
      processo for inicializado para calcular cada elemento de $X$?
  \end{enumerate}
\end{frame}

\begin{frame}[fragile]{Solução: compartilhar memória}
  \large
  Função mmap: retorna um ponteiro para uma área de memória que é compartilhada
  entre processos. Deve ser alocada de forma semelhante a um malloc:
  \begin{verbatim}
  int protection = PROT_READ | PROT_WRITE;
  int visibility = MAP_SHARED | MAP_ANON;

  /* Criar area de memoria compartilhada */
  int *b;
  b = (int*) mmap(NULL, sizeof(int)*100,
                protection, visibility, 0, 0);
  \end{verbatim}

  Leitura detalhada sobre o que significa cada parâmetro: \textsc{man mmap}
\end{frame}


\begin{frame}[fragile]{Demonstração: multiprocesso com e sem mmap}
  \centering
  \Large
  Veja a demonstração de multiprocesso com e sem mmap.
\end{frame}

\begin{frame}[fragile]{Memory Maps}
  \centering
  \Large
  Em quais dessas aplicações seria desejável compartilhar memória entre
  processos com mmap:
  \begin{enumerate}
    \item Realizar cálculos longos de forma distribuída
    \item Responder a requisições num webserver
    \item Encadear diversos processadores (como plugins) de imagem diferentes
  \end{enumerate}
\end{frame}

\begin{frame}[fragile]{Problema}
  \centering
  \large
  Na demonstração de código que você verá, identifique:
  \begin{enumerate}
  \item Qual é o padrão de design usado (produtor-consumidor, dispatcher ou
    tarefas paralelas)?
  \item O que o programa deveria fazer?
  \item O que o programa está, realmente, fazendo?
  \item Que tipo de funcinalidade poderia corrigir esse problema?
  \end{enumerate}
\end{frame}

\begin{frame}[fragile]{Idéia: variável de controle de acesso}
  \centering
  \large
  Na demonstração de código que você verá, identifique:
  \begin{enumerate}
  \item Qual é a variável que controla o acesso aos recursos compartilhados?
  \item Por que o processo p2 não fica em loop infinito na linha
    \texttt{while(N==0);}?
  \end{enumerate}
\end{frame}



\end{document}
