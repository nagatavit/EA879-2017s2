\documentclass{beamer}
%
% Choose how your presentation looks.
%
% For more themes, color themes and font themes, see:
% http://deic.uab.es/~iblanes/beamer_gallery/index_by_theme.html
%
\mode<presentation>
{
  \usetheme{Madrid}      % or try Darmstadt, Madrid, Warsaw, ...
  \usecolortheme{default} % or try albatross, beaver, crane, ...
  \usefonttheme{default}  % or try serif, structurebold, ...
  \setbeamertemplate{navigation symbols}{}
  \setbeamertemplate{caption}[numbered]
}
\usepackage[normalem]{ulem}
\usepackage[english]{babel}
\usepackage[utf8x]{inputenc}
\usepackage{graphicx}
\usepackage{array}

\title[T1-Trabalho 1]{EA879 -- Introdução ao Software Básico\\Trabalho 1}
\author{Tiago F. Tavares}
\institute{FEEC -- UNICAMP}
\date{12/setembro/2017}

\begin{document}

\begin{frame}
  \titlepage
\end{frame}

% Uncomment these lines for an automatically generated outline.
%\begin{frame}{Outline}
%  \tableofcontents
%\end{frame}

\section{Introdução}

\begin{frame}{Objetivos}
  \Large
  \begin{itemize}
    \item Analisar textos científicos
    \item Identificar falhas em raciocínios científicos
  \end{itemize}

Hoje, excepcionalmente, tente se sentar em grupos de pessoas que você não conhece,
  evitando sentar-se com membros do seu grupo de trabalho. Se precisar, tome um
  tempo para se apresentar.
\end{frame}

\begin{frame}[fragile]{Formato}
  \centering
  \Large
  Nesta aula, você deve ter, em mãos, uma cópia de seu relatório. Verifique,
  inicialmente, se o formato está correto.
  \begin{enumerate}
    \item Seu relatório tem título?
    \item Todos os autores estão identificados?
    \item O relatório está dentro do limite de páginas (1)?
  \end{enumerate}
  O relatório que você tem em mãos hoje não será lido pelo professor (a não ser
  que você peça), então fique à vontade para escrever o que quiser.
\end{frame}

\begin{frame}[fragile]{Reflexão (parte 1)}
  \centering
  \Large
  No verso do seu relatório:
  \begin{enumerate}
    \item Na lateral esquerda, faça uma lista com o nome de todos os membros do
      grupo
    \item Na lateral direita, faça uma lista de todos os objetivos intermediários
    que foram cumpridos para completar o projeto (por exemplo: ``projetar
    gramática livre de contexo'', ``implementar código no
    yacc'', ``redigir relatório'' e assim por diante).

  \end{enumerate}

\end{frame}


\begin{frame}[fragile]{Reflexão (parte 2)}
  \centering
  \Large
  No verso do seu relatório, relacione pessoas aos objetivos que foram
  cumpridos, usando setas diferentes (cores, tracejado, etc.) para cada
  categoria de participação:
  \begin{enumerate}
    \item Fez sozinho(a)
    \item Participou ativamente
    \item Acompanhou a discussão
  \end{enumerate}

\end{frame}


\begin{frame}[fragile]{Reflexão (parte 3)}
  \centering
  \Large
  Discuta com seu grupo (de aula):
  \begin{enumerate}
    \item As tarefas do seu grupo ficaram bem distribuídas?
    \item Quais fatores (externos e internos ao grupo de trabalho) contribuíram
        para isso?
  \end{enumerate}
\end{frame}

\begin{frame}[fragile]{Leitura de texto}
  \centering
  \large
  Passe seu texto para a pessoa que está à sua esquerda. Após esse processo,
  cada membro do grupo deve ter em mãos um relatório do qual não é autor. Leia o
  texto, identificando se existe:
  \begin{enumerate}
  \item Um breve texto introdutório que discute do que o texto se trata?
  \item Uma demonstração das funcionalidades implementadas
  \item Análise, comparando a idéia de usar comandos específicos da linguagem
    criada pelo grupo com uma  aplicação equivalente, usando alguma biblioteca
      de linguagem de propósito geral.
  \end{enumerate}

  Circule no texto as partes identificadas e as rotule como ``contexto'',
  ``demonstração'' e ``análise''. Há partes do texto sem função clara?
\end{frame}

\begin{frame}[fragile]{Gramática}
  \centering
  \large
  No texto todo, sublinhe os seguintes aspectos:
  \begin{enumerate}
    \item Erros de escrita/gramática (todos que puder achar).
    \item Vícios de linguagem:
      \begin{enumerate}
        \item Sujeito indeterminado ou reflexivo (``faz-se'', ``propõe-se'')
        \item Tempo verbal futuro, pretérito mais-que-perfeito, mesóclise
        \item Voz passiva com conjugação errada (``foi feit\textbf{o} \textbf{a
          compilação}'')
      \end{enumerate}
  \end{enumerate}
\end{frame}


\begin{frame}[fragile]{Fluxo de idéias}
  \centering
  \large
  Agora, releia o texto com foco no fluxo de idéias.
  \begin{enumerate}
    \item Para cada frase:
      \begin{enumerate}
        \item Identifique os conceitos que aparecem na frase.
        \item Para cada conceito, se ele aparece pela primeira vez no texto,
          verifique se ele aparece de:
          \begin{enumerate}
            \item Senso comum (ex: $F = MA$)
            \item Uma citação ([4])
            \item Um raciocínio (... portanto, corpos no vácuo caem com a mesma
              aceleração)
            \item Uma observação direta (o pêndulo tem massa 1kg)
          \end{enumerate}
        \item Se não aparece, circule o conceito, mostrando que a origem dele
          precisa ser melhor explicada.
       \end{enumerate}
   \end{enumerate}
\end{frame}

\begin{frame}[fragile]{Relevância 1}
  \centering
  \large
  Releia novamente o texto. Passe um traço sobre todas as passagens que são
  irrelevantes (não-essenciais) para a construção da idéia central do documento,
  mesmo que sejam corretas e adequadamente embasadas. Também passe um traço
  sobre hipérboles. Exemplos:
  \begin{itemize}
    \item Portanto, a potência é de 30 Watts, \sout{unidade batizada em homenagem ao engenheiro
      James Watt [1], inventor do regulador de velocidade de máquinas a vapor [2]}
    \item \sout{O algoritmo foi implementado em MatLab 2014.}
    \item \sout{A construção de uma linguagem de programação é parte fundamental da
      formação de um engenheiro.}
    \item \sout{O experimento teve sucesso.}
  \end{itemize}
\end{frame}

\begin{frame}[fragile]{Conjunções}
  \centering
  \large
  No texto todo, encontre as conjunções -- palavras como ``mas'', ``e'',
  ``portanto'', etc. que dão sentido às conexões de frases. Identifique (e
  sublinhe) todos os conectivos que não estão adequados às idéias mostradas. Por
  exemplo:
  \begin{enumerate}
    \item Compiladores são relevantes, portanto fizemos um trabalho de
      compiladores (a relação de implicação não deveria existir).
    \item Imagens são interessantes, assim, faz-se necessário uma linguagem
      específica para processamento (não existe relação de implicação, além do
      sujeiro reflexivo e da conjugação errada da voz passiva).
  \end{enumerate}
\end{frame}

\begin{frame}[fragile]{Idéia geral}
  \centering
  \large
  Para cada parte do texto (motivação, demonstração, discussão), verifique se
  ela cumpre sua função. Desconsidere, para isso, a gramática e a escrita, a não
  ser que elas atrapalhem muito a compreensão do texto. Atribua um conceito a
  cada parte:

  \begin{itemize}
    \item A -- cumpre a função e está bem embasada
    \item B -- alguns aspectos não estão bem embasados ou está parcialmente
      incompleto, mas isso não atrapalha o texto como um todo
    \item C -- está ok (``dá para entender''), mas tem muitas idéias mal
      embasadas ou mal conectadas, e isso atrapalha o texto como um todo
    \item D -- não cumpre bem a função
  \end{itemize}

\end{frame}

\begin{frame}[fragile]{Idéias interessantes}
  \centering
  \large
  No texto, marque (use um emoji) as conclusões que estão bem embasadas, e que
  são relevantes para a disciplina (ex: ``portanto, essa solução poderia ser
  usada para otimizar processos de produção de código'' -- se for uma frase
  corretamente embasada)
\end{frame}

\begin{frame}[fragile]{Nota -- predição}
  \centering
  \large
  Atribua notas ao texto. Todo texto começa com 10.
  \begin{itemize}
    \item Nas avaliações de seção: $-1$ para cada nota B, $-2$ para cada C, $-3$ para
  cada nota D.
    \item $-0.5$ para cada conjunção mal empregada (que foi sublinhada)
    \item $-0.1$ para cada erro de gramática ou de escrita (que foi sublinhado)
    \item $-1$ para cada conceito que aparece sem justificativa (que foi circulado)
    \item $+2$ para cada idéia interessante (que foi marcada com emoji)
  \end{itemize}
\end{frame}

\begin{frame}[fragile]{Devolução}
  \centering
  \large
  Devolva o relatório, com marcações, ao seu portador original. Após receber seu
  relatório de volta, reuna-se com o seu grupo de trabalho e discuta brevemente
  sobre o que foi observado, e se vocês concordam com as observações dos
  revisores.
\end{frame}

\begin{frame}[fragile]{Discussão}
  \centering
  \large
  Discuta com seu grupo:
  \begin{enumerate}
    \item Como você se sentiu ao revisar o relatório de outro grupo?
    \item Quais foram as maiores dificuldades encontradas no processo?
    \item Você acha que sua contribuição foi relevante para o relatório do outro
      grupo?
    \item Você acha que a contribuição dos revisores foi relevante para você?
    \item Você aprendeu ou descobriu algo relevante ao ler o relatório de outro
      grupo?
    \item Em que outras situações um processo de revisão como esse poderia ser
      relevante?
  \end{enumerate}
\end{frame}

\end{document}
